\documentclass{ees}

\begin{document}

\eesTitlePage

\eesCriticalReport{
  – & –   & org & All bass figures have been added by the editor. \\
  3 & –   & cor & According to \A1, this part may be played by trombas or horns. \\
  4 & 21  & ob 1, vl 1 & grace note added by editor \\
    & 34  & ob 1 & 2nd \halfNote\ in \A1: \sharp f″2 \\
    & 34  & vl   & 3rd \quarterNote\ in \A1: a′16–\sharp f″16–a′16–\sharp f″16 \\
    & 90  & ob 1, vl 1 & grace note added by editor \\
    & 103 & vl   & 3rd \quarterNote\ in \A1: d″16–b″16–d″16–b″16 \\
    & 120 & ob 1, vl 1 & grace note added by editor \\
    & 134 & S   & 3rd \quarterNote\ in \A1: c″4 \\
}



\eesToc{
\textit{Soloists}\\[1ex]
Levita (Soprano)\\
Filia Petri, Sacerdos 1:mus (Tenore)\\
Sacerdos 2:dus (Basso)
\par\bigskip

\begin{movement}{ansomnio}
  \voice[Filia Petri]
  An somnio?
  Anne vigilo?
  Quis dulcis meas concentus aures occupat?
  Solymae decus insigne templum,
  in Spiritu quondam pio visum Prophetae,
  hac nocte quoque visum est mihi.

  \voice[Levita]
  Dilecta mater!
  Somnium o gratum nimis habui!
  Viginti quinque, repetita in libro Ezechielis saepius,
  vidi aureis expressa numeris,
  ubi de templo Israel et civitate sancta fit mentio.
  Auguror, felicitatis plurimum numero sacro hoc contineri.
  Sed sacerdotes duo,
  quibus ante somnium istud exposui, advolant,
  et tibi, mihique nuntiaturi bona.

  \voice[Sacerdos 1:mus]
  Fecunda salve Sponsa virginei optimi Sponsi,
  viginti quinque qui jamjam integros explevit annos,
  gereret ut curam tui.

  \voice[Filia Petri]
  Jam capio dulce somnium, donum Dei.
  Propheta quod de sede Solymarum sacra
  et civitate dixit in Spiritu,
  id in his in veritate operatus est,
  per Praesulem muris Deus.
  Cor gaudii plenum exsilit.
\end{movement}

\begin{movement}{ansomnio}
  \voice[Filia Petri]
  Quae planctus in ruinis
  vetustis confirmavit,
  requirens repentinis
  haec curis restauravit,
  o me beatam Numinis
  faventis sub clementia,
  o me renatam vigilis
  sub Beda providentia.
\end{movement}

\begin{movement}{gauderebecca}
  \voice[Sacerdos 2:dus]
  Gaude Rebecca sterilis, et tristis diu!
  ridere jam cum fertili hac sponsa vales;
  gravi in semecta filium nam tibi Deus tribuit
  parentis propriae ut fieret Pater.
  Speciosa facta es amica sub tanto Patre
  qui bis decem atque quinque connubii sui annis
  vetustam transtulit faciem
  et nigrum tersit colorem,
  filiosque tibi novos peperit,
  amoris germina et testes sui.

  \voice[Levita]
  Hos inter et ego gratia indignus fruor,
  et me beatum reputo sub tanto duce,
  qui primus in Virtute commissum gregem
  secum per arctam promovet coeli viam.
  Quis huic Parenti gratias dignas satis umquam rependat?
  Aut quis enumeret bona collata matri,
  cui novam vitam dedit?
  Loquantur arae, templa, picturae,
  foris intusque muri beneficum
  in Beda Patrem alta loquantur voce
  et aeternas pio hoc pro Parente gratias referant DEo.
\end{movement}

\begin{movement}{telaudamus}
  \voice[Coro]
  Te laudamus o Supremum
  et tremendum coeli Numen!
  quod per Bedam nigrae demum
  suscitasti Sponsae lumen.

  \voice[Filia Petri]
  Me amavit et ornavit,

  \voice[Levita]
  me dilexit et direxit,

  \voice[Sacerdos 1:mus et 2:dus]
  per viginti quinque Soles
  Patris instar nos ut proles,
  sustentavit optime,
  gubernavit provide.

  \voice[Coro]
  Hinc o Numen te laudamus,
  grata mente praedicamus,
  ah, conserva quod rogamus
  nobis Patrem quem amamus.
\end{movement}

\begin{movement}{adeste}
  \voice[Filia Petri]
  Adeste!
  quotquot estis in templo Dei,
  populumque convocate,
  ut hodiernam diem, nobis beatam,
  civibusque bonae indolis optatam,
  amoenis musices celebret modis.
  Haec est dies, quam Dominus ad nostrum bonum,
  ad nostra fecit vota: laetemur in ea.
\end{movement}

% \begin{movement}{}
%   \voice[]
% \end{movement}

% \begin{movement}{}
%   \voice[]
% \end{movement}

% \begin{movement}{}
%   \voice[]
% \end{movement}
}

\eesScore

\end{document}
